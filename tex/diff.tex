\documentclass[12pt,a4paper]{scrartcl}
\usepackage[utf8]{inputenc}
\usepackage[english,russian]{babel}
\usepackage{indentfirst}
\usepackage{misccorr}
\usepackage{graphicx}
\usepackage{mathtools}
\DeclareMathOperator{\arccosh}{arccosh}
\DeclareMathOperator{\arcsinh}{arcsinh}
\DeclareMathOperator{\arctanh}{arctanh}
\DeclareMathOperator{\e}{e}
\DeclareMathOperator{\tangent}{tangent}
\begin{document}
\section{Ваша изначальная функция:} 
 \begin{equation} f(x, y, z) = {(\ln{({({x}+{y})}+{z})})}\end{equation}
\section{Значение функции в точке.}
\noindent \textbf {Значения переменных при рассчете значения функции:}
\begin{enumerate}
	\item x = 10,000000
	\item y = 25,000000
\end{enumerate}
\noindent \textbf {Значение функции в заданной точке:}
 \begin{equation} f(10, 25, z) = {(\ln{({35}+{z})})}\end{equation}

\end{document}